\documentstyle[times,ilc]{article}

\begin{document}

%Put here the title of your abstract
\title{On the use of sub-mesoscale tracer information for the control of ocean circulations}

%Put here the list of authors and their affiliation
\author{L. Gaultier, J. Verron, P. Brasseur and J.-M. Brankart}
\affiliation{LEGI/CNRS, France}

%If you have authors from different institutions, please adapt one of
%these examples
%
%Example 1
%---------
%\author{Robin Muensch$^1$, Richard Dewey$^2$ and John Gunn$^1$}
%\affiliation{$^1$Earth and Space Research, USA \\
%$^2$University of Victoria, Canada}
%
%Example 2
%---------
%\author{Hans Burchard}
%\affiliation{Space Application Institute, Joint Research Center, Italy}
%\author{Ole Peterson}
%\affiliation{Danish Hydraulic Institute, Denmark}

%Do not forget to include this line
\maketitle

%Type in the core of your abstract
  
In the ocean, turbulent dynamics encompasses a wide range of scales and is characterized by high energy levels including at smaller scales such as the so-called sub-mesoscales (e. g. Klein et al, 2010). Ocean observations from space include chlorophyll and sea surface temperature (SST) measurements that are of great interest for synoptic ocean studies. These tracer fields reveal also a clear signal at the sub-mesoscales in most regions of the world ocean. There is more and more evidence that the understanding and modelling of the physico-biogeochemical behavior of the ocean requires to know more about those various scales, to understand their role and interactions and to adequately represent or parameterize the mesoscales and the sub-mesoscales in ocean models. In this regard, tracers are of great value as they are the only fine resolution observations that provide insights on those scales and in particular on the sub-mesoscales.\\ 
\ 

In this work, our objective is to explore the feasibility of using tracer information at the sub-mesoscales to possibly control ocean dynamic fields. In order to do so, we choose an image data assimilation strategy in which a cost-function is built that aims at minimizing the misfits between some image of sub-mesoscale flow structure and tracer images. In the present work we have explored the extent to which the Finite-Size Lyapunov Exponent (FSLE) can be considered as a proxy characterizing the sub-mesoscale flow structure. The FSLEs represent the stretching of the geophysical fluids (Lehahn et al, 2007) and extremal values can be seen as frontal barriers in the studied fluid. The choice of Lyapunov Exponents as a proxy for tracers is motivated by d'Ovidio et al (2004), where similar patterns between tracers and FSLE images are evidenced.\\ 
\ 

A prerequisite to the study is that the relation between the ocean dynamics and FSLE can be inverted, that is to say that the sub-mesoscale information transmitted through the intermediate FSLE proxy is effective in controlling the ocean system.\\ 
\ 
This assumption is tested in a 10� by 10� regional piece of ocean in the North East Atlantic. The ocean dynamical variable to be corrected is the mesoscale altimetric velocity field. Therefore, the sensitivity of FSLE horizontal patterns to velocity errors is investigated. To do so, a Gaussian velocity error field is created using up to fifteen years of altimetric data. A cost function is then defined to measure the misfit between the Lyapunov exponents computed using velocities with errors and the Lyapunov exponents derived from a 'true' (error free) velocity. It is found that a global minimum can be identified in this cost function proving that the inversion of FSLE is feasible. \\ 
\ 

These results show the feasibility of assimilating sub-mesoscale into ocean models for the control of mesoscale dynamics and larger scales. But the calculations must be extended and generalized. Use of SST tracer data from a numerical simulation also proved to be successful. A further step must be the use of actual sub-mesoscale images from chlorophyll or SST. The difficulty remains in identifying properly frontal structures in tracer images so that an equivalent of FSLE image using a real snapshot of a tracer is computed. Other proxies than Lyapunov exponents may also be thought of.


%If you want to include acknowledgments, please use this environment :
%\begin{acknowledgments} The authors are grateful to ...
%\end{acknowledgments}

%If you include some references, please use the environment {thebibliography}
%and introduce the reference in the text with the \cite command.
%
% Example
% -------
%
%    ... \cite{Lev}...
%
%\begin{thebibliography}{}
%\bibitem[{\em Levine et al.,} 1985]{Lev} Levine, M.D., Paulson, C.A.  and
%Morison J.H., 1985.  Internal waves in the Artic Ocean :  A comparison with
%lower-latitude observations.  {\it J.  Phys.  Oceanog.}, 15, 800-809.
%
%\bibitem[{\em Gregg,} 1989]{Gregg} 
%Gregg, M.C., 1989.  Scaling turbulent diffusion in the thermocline
%{\it J.  Geophys. Res.}, 94, 9686-9698.
%
%\end{thebibliography}


%You can also include your full address at the end of the abstract.  Please 
%insert a blank line between successive addresses
%\begin{addresses}
%GeoHydrodynamics and Environment Research Lab., University of Liege, Sart Tilman
%B5, B-4000 Liege.
%\end{addresses}


\end{document}

%

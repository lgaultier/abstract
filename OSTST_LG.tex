ON THE USE OF HIGH RESOLUTION TRACER IMAGE AND ALTRIMETRIC FIELD FOR THE CONTROL OF OCEAN CIRCULATIONS
Lucile Gaultier, Jacques Verron, Pierre Brasseur, Jean-Michel Brankart
LEGI/CNRS, Grenoble, France


In the ocean, turbulent dynamics encompasses a wide range of scales. 
Ocean mesocale eddies are essential components of the ocean circulation which is currently best observed by altimetric satellites. 
Additionally, high resolution sensors of bio-geochemical tracers such as the Sea Surface Temperature or the Ocean Color reveal even smaller structures at the sub-mesoscale, which are not detected by altimetric satellites. 
The role of the sub-mesoscale in the ocean happens to be very important for the dynamic and the energy level at larger scale (e.g. Klein et al, 2010). 
Therefore, we must benefit from the two types of complementary observations (mesoscale dynamic and sub-mesoscale tracer image) to refine the estimation of the ocean circulation.


The goal of this study is to explore the feasibility of using tracer information at the submeso-scales to possibly control ocean dynamic fields that emerge from altimeter data analysis at larger scales. 
In order to do so, we choose an image data assimilation strategy in which a cost-function is built that aims at minimizing the misfits between some image of submeso-scale flow structure and tracer images. 
In the present work we have explored the extent to which the Finite-Size Lyapunov Exponents (FSLE) can be considered as a proxy characterizing the submeso-scale flow structure. 
The choice of Lyapunov Exponents as a proxy for tracers is motivated by d'Ovidio et al (2004), where similar patterns between tracers and FSLE images are evidenced. 

A prerequisite to the study is that the relation between the ocean dynamics and FSLE can be inverted, that is to say that the submeso-scale information transmitted through the intermediate FSLE proxy is effective in controlling the ocean system. 
This assumption has been sucessfully tested on several regional pieces of the ocean.
Using a strategy similar to the one used in Data Assimilation, the sensitivity of FSLE horizontal patterns to velocity errors is investigated. 
To do so, a Gaussian velocity error field is created using up to fifteen years of altimetric data. 
A cost function is then defined to measure the misfit between the Lyapunov exponents computed using velocities with errors and the Lyapunov exponents derived from a `true' (error free) velocity. 
It is found that a global minimum can be identified in this cost function proving that the inversion of FSLE is feasible. 
The next step is the inversion of sub-mesoscale tracer information to correct a mesoscale altimetric field using real observation (from AVISO for the velocity and from MODIS sensor for the tracer). 
The ocean dynamical variable to be corrected is the mesoscale altimetric velocity field using a high resolution tracer image. 
The strategy is quite similar to the one used to invert FSLE. The altimetric velocity field and its derived FSLE are assumed to contain errors and are corrected using the high resolution tracer image. 
The cost function measures the misfit between the FSLE derived from the altimetric velocity and 
the high resolution tracer image. 
Several test case have been studied to prove the feasibility and the success of the inversion of sub-mesoscale tracer information to correct a mesoscale altimetric velocity field.

These results show the feasibility of assimilating tracer submeso-scales into ocean models for the control of mesoscale dynamics and larger scales as deduced from altimetry.
The conjoint use of high resolution tracer data and altimetric data enable to improve the description of the ocean.

